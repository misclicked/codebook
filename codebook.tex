\documentclass[10pt,twocolumn,oneside]{article}
\setlength{\columnsep}{10pt}                    %兩欄模式的間距
\setlength{\columnseprule}{0pt}                 %兩欄模式間格線粗細

\usepackage{amsthm}								%定義,例題
\usepackage{amssymb}
\usepackage{fontspec}							%設定字體
\usepackage{color}
\usepackage[x11names]{xcolor}
\usepackage{listings}							%顯示code用的
\usepackage{fancyhdr}							%設定頁首頁尾
\usepackage{graphicx}							%Graphic
\usepackage{enumerate}
\usepackage{titlesec}
\usepackage{amsmath}
\usepackage[CheckSingle, CJKmath]{xeCJK}
 \usepackage{CJKulem}

\usepackage{amsmath, courier, listings, fancyhdr, graphicx}
\topmargin=0pt
\headsep=5pt
\textheight=740pt
\footskip=0pt
\voffset=-50pt
\textwidth=545pt
\marginparsep=0pt
\marginparwidth=0pt
\marginparpush=0pt
\oddsidemargin=0pt
\evensidemargin=0pt
\hoffset=-42pt

%\renewcommand\listfigurename{圖目錄}
%\renewcommand\listtablename{表目錄}

%%%%%%%%%%%%%%%%%%%%%%%%%%%%%

\setmainfont{YaHei.Consolas.1.12.ttf}
%\setmonofont{Ubuntu Mono}
\setmonofont{YaHei.Consolas.1.12.ttf}
\setCJKmainfont{Noto Sans CJK TC}
\XeTeXlinebreaklocale "zh"						%中文自動換行
\XeTeXlinebreakskip = 0pt plus 1pt				%設定段落之間的距離
\setcounter{secnumdepth}{3}						%目錄顯示第三層

%%%%%%%%%%%%%%%%%%%%%%%%%%%%%
\makeatletter
\lst@CCPutMacro\lst@ProcessOther {"2D}{\lst@ttfamily{-{}}{-{}}}
\@empty\z@\@empty
\makeatother
\lstset{										% Code顯示
    language=C++,									% the language of the code
    basicstyle=\footnotesize\ttfamily, 					% the size of the fonts that are used for the code
    %numbers=left,									% where to put the line-numbers
    numberstyle=\footnotesize,					% the size of the fonts that are used for the line-numbers
    stepnumber=1,									% the step between two line-numbers. If it's 1, each line  will be numbered
    numbersep=5pt,									% how far the line-numbers are from the code
    backgroundcolor=\color{white},				% choose the background color. You must add \usepackage{color}
    showspaces=false,								% show spaces adding particular underscores
    showstringspaces=false,						% underline spaces within strings
    showtabs=false,								% show tabs within strings adding particular underscores
    frame=false,										% adds a frame around the code
    tabsize=2,										% sets default tabsize to 2 spaces
    captionpos=b,									% sets the caption-position to bottom
    breaklines=true,								% sets automatic line breaking
    breakatwhitespace=false,						% sets if automatic breaks should only happen at whitespace
    escapeinside={\%*}{*)},						% if you want to add a comment within your code
    morekeywords={*},								% if you want to add more keywords to the set
    keywordstyle=\bfseries\color{Blue1},
    commentstyle=\itshape\color{Red4},
    stringstyle=\itshape\color{Green4},
}


\begin{document}
\pagestyle{fancy}
\fancyfoot{}
%\fancyfoot[R]{\includegraphics[width=20pt]{ironwood.jpg}}
\fancyhead[C]{National Cheng Kung University}
\fancyhead[L]{Ted本人}
\fancyhead[R]{(\today) \thepage}
\renewcommand{\headrulewidth}{0.4pt}
\renewcommand{\contentsname}{Contents}

\scriptsize
\tableofcontents
\section{Basic}

\subsection{/.vimrc}
\lstinputlisting{Basic/vimrc}
\subsection{default code}
\lstinputlisting{Basic/default.cpp}
\subsection{debug list}
\lstinputlisting{Basic/bug_list.txt}

\section{Geometry}

\subsection{2D Point Template}
\lstinputlisting{Geometry/2Dpoint.cpp}
\subsection{外心 Circumcentre}
\lstinputlisting{Geometry/circumcentre.cpp}
\subsection{Convex Hull}
\lstinputlisting{Geometry/ConvexHull.cpp}
\subsection{半平面交}
\lstinputlisting{Geometry/half_plane_intersection.cpp}
\subsection{圓交}
\lstinputlisting{Geometry/Intersection_of_two_circle.cpp}
\subsection{線段交}
\lstinputlisting{Geometry/Intersection_of_two_lines.cpp}
\subsection{Smallest Covering Circle}
\lstinputlisting{Geometry/Smallest_Circle.cpp}

\section{Mathmatics}

\subsection{ax+by=gcd(a,b)}
\lstinputlisting{Math/ax+by=gcd.cpp}
\subsection{BigInt}
\lstinputlisting{Math/Bigint.cpp}
\subsection{FFT}
\lstinputlisting{Math/fft.cpp}
\subsection{FWHT}
\lstinputlisting{Math/FWHT.cpp}
\subsection{GaussElimination}
\lstinputlisting{Math/GaussElimination.cpp}
\subsection{Inverse}
\lstinputlisting{Math/Inverse.cpp}
\subsection{LinearPrime}
\lstinputlisting{Math/LinearPrime.cpp}
\subsection{Pollard's rho}
\lstinputlisting{Math/pollardRho.cpp}
\subsection{數論基本工具}
\lstinputlisting{Math/number_tool.cpp}
\subsection{Mobius}
\lstinputlisting{Math/Mobius.cpp}
\subsection{Simplex}
\lstinputlisting{Math/Simplex.cpp}
\subsection{SG}
\lstinputlisting{Math/Sprague-Grundy.cpp}
\subsection{Theorem}
\lstinputlisting{Math/theorem.cpp}


\section{Graph}

\subsection{BCC}
\lstinputlisting{Graph/BCC_edge.cpp}
\subsection{Dijkstra}
\lstinputlisting{Graph/Dijkstra.cpp}
\subsection{Theorm - Domination}
\lstinputlisting{Graph/Domination.txt}
\subsection{Strongly Connected Component(SCC)}
\subsection{DominatorTree}
\lstinputlisting{Graph/DominatorTree.cpp}
\lstinputlisting{Graph/Kosaraju_SCC.cpp}
\subsection{Manhattan MST}
\lstinputlisting{Graph/ManhattanMST.cpp}
\subsection{Hungarian}
\lstinputlisting{Graph/Matching/Hungarian.cpp}
\subsection{KM}
\lstinputlisting{Graph/Matching/Kuhn_Munkres.cpp}
\subsection{Theorm - Matching}
\lstinputlisting{Graph/Matching/Matching.txt}
\subsection{Maximum General Matching}
\lstinputlisting{Graph/Matching/Maximum_General_Matching.cpp}
\subsection{Minimum General Weighted Matching}
\lstinputlisting{Graph/Matching/Minimum_General_Weighted_Matching.cpp}
\subsection{Maximum Clique}
\lstinputlisting{Graph/MaximumClique.cpp}
\subsection{Steiner Tree}
\lstinputlisting{Graph/MinimumSteinerTree.cpp}
\subsection{最小平均環}
\lstinputlisting{Graph/Min_mean_cycle.cpp}
\subsection{SchreierSims}
\lstinputlisting{Graph/SchreierSims.cpp}
\subsection{Tarjan}
\lstinputlisting{Graph/Tarjan.cpp}
\subsection{2-SAT}
\lstinputlisting{Graph/TwoSAT.cpp}


\section{Data Structure}

\subsection{Sparse Table}
\lstinputlisting{DataStructure/SparseTable.h}
\subsection{Segment Tree}
\lstinputlisting{DataStructure/Seg.cpp}


\section{String}

\subsection{KMP}
\lstinputlisting{String/KMP.h}


\section{Dark Code}

\subsection{輸入優化}
\lstinputlisting{DarkCode/IO_optimization.cpp}
\subsection{PBDS}
\lstinputlisting{DarkCode/PBDS.cpp}

\section{Search}

\subsection{LIS}
\lstinputlisting{Search/LIS.cpp}
\subsection{Merge sort}
\lstinputlisting{Search/MergeSort.cpp}


\section{Others}

\subsection{數位統計}
\lstinputlisting{Other/DigitCounting.cpp}
\subsection{1D/1D dp 優化}
\lstinputlisting{Other/Dp1D1D.cpp}
\subsection{Theorm - DP optimization}
\lstinputlisting{Other/DP-optimization.txt}
\subsection{Mo's algorithm}
\lstinputlisting{Other/Mo_algorithm.cpp}



\section{Persistence}



\end{document}
